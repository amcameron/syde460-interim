% vim: tw=80 et sw=4
\documentclass{sydeStyle}
\usepackage{amsmath}
\usepackage{amssymb}
\usepackage{graphicx}

\coursecode{461}
\prof{Professor Stashuk}
\title{
Design of a Hang-Glider Autonomous Controller\\
Interim Report
}
\date{\today}

\author{Andrew Cameron, 20252410}
\authorthree{Peter Lobsinger, 20195824}

\begin{document}

\maketitle

\chapter{Introduction}
% This should introduce the topic in a manner understandable by anyone with a
% reasonable technical background (i.e. any 1st year engineering student should
% be able to understand the problem). Identify the need or problem that you want
% to address and why you want to address it. Indicate why and to whom the
% problem is important.  Background information should motivate the need for the
% design and show the nature, extent or significance of the problem you have
% defined.  This can be done, for example, by identifying inadequacies in
% existing designs, demonstrating an industry need for such an implementation,
% or reviewing existing research on the topic.  These three aspects
% (Introduction, Background and Problem Statement) can be organized in separate
% sections, or all together, etc. in order to suit the readability for your
% particular workshop.

The Waterloo Rocketry Team (WRT) is designing and building a rocket to compete
in the 2011 Intercollegiate Rocket Engineering Competition. One aspect of the
competition is a payload which will be assessed on functionality and
innovation.\cite{IREC} To satisfy this component, the WRT has opted to create a
several alternative payloads one of which is a glider to be deployed at the apex
of the rocket's flight.

The concept for the glider is a springloaded parasev-type craft, 
consisting of a control mass supended bellow a triangular wing composed of
sailcloth or other flexible material, similar to a conventional
hang-glider.\cite{wiki:parasev} The glider must fit within the payload area of
the rocket and will therefore be no longer than 50cm in length.

The mechanical design of a glider suitable for this task was the subject of a
design project by 4A Mechanical Engineering student, Gandhali Joshi. However,
the control of this craft has not yet been considered.

Control of the craft is imperative as failure to recove any component of
the rocket or payload will result in disqualification from the
competition.\cite{IREC} It is therefor the objective of this design project to
design a control system for this glider to allow for ease of recoverability.

\chapter{Objectives}
% As with the Design Plan, objectives must be specific and demonstrated to be
% realistic.  The objectives should not have significantly changed from the
% Design Plan; however, you may want to rework them to ensure that they fit in
% with your current progress.

The deployment strategy, managed by the rocketry team, should leave the glider
clear of the rocket and deploying with an attitude suitable to flight. This is
where the task of the controller begins.

\section{Maintain Flight}

When the glider has acheived stable flight, it must remain in stable flight
until it reaches the ground. Meteorological conditions may make this infeasible.
It is therefor necessary to temper this requirement in moderate winds. For the
purposes of this project, moderate wind has been defined as having a ground
speed no greater than 30km/h, at the upper end of "Moderate Breeze" on the
Beaufort scale \cite{wiki:beaufort}.

\section{Fly to Target}

The main objective of the controller is to seek out and arrive at a
pre-determined position on the ground, to allow for the easy recovery of the
craft upon landing. This is only feasible for certain sets of initial conditions
and meteorological conditions.

The rocket provides some guidance as to the initial conditions, while the
meteorological conditions have been limited in the objective above.

Specifically, the bounds on the initial conditions are: an altitude of between 2
and 5 km above ground level, corresponding to a wide range about the maximum
apex of the rocket flight; lateral displacement from target of no more than 1km,
corresponding to the worst case estimate of the rockets trajectory; between 0
and 9.8m/s downward velocity, corresponding to up to 1 second of freefall before
deployment; between 0.5 and 3m/s ground speed, corresponding to estimates
of rocket flight apex velocity; any attitude that will give an angle of attack
of between 0 and 30 degrees, given the initial velocities, corresponding to the
assumption that the deployment procedure will position the craft in an
orientation favourable to flight; and zero spin rates, corresponding to the
assumption that freefall will be dominated by translational, not rotational,
motions.

\section{Land Safely}

The craft is intended to be reused, the craft must not be destroyed
upon landing. Therefor, impact forces, and impact velocities causing these must
be limited. As the craft is unmanned, moderate impact shocks upon landing are
acceptable.

\chapter{Methodology}
% At this point in the design process, your design methodology should be
% sufficiently developed so that the path to achieve the objectives is clear.
% Criteria (measurable qualities) and constraints (hard decisions) should be
% provided to guide the design process and motivate preferred solutions.  The
% extent of the use of criteria and constraints is project dependent.  The
% design group should provide sufficient information in this section to fully
% justify the methods used in the future of the workshop project.  If two or
% more potential solutions are being evaluated, then a decision matrix should be
% (eventually) utilized.

% NOT IN INTRO
% This design was the subject of much
% investigation by NASA.\cite{Rogallo1960}\cite{Layton1963} However, the goal of these
% investigations was manned flight; as opposed to fully-autonomous flight, which
% would be the case here.

\chapter{Timeline}
% The timeline provided with the Design Plan should be updated to indicate the
% current state of the project.  A concise description indicating how well the
% group is following their projected time-line should be placed in an appendix
% (along with the timeline).

\bibliography

\end{document}
