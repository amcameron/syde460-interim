% vim: tw=80 et sw=4
\documentclass{sydeStyle}
\usepackage{amsmath}
\usepackage{amssymb}
\usepackage{graphicx}

\coursecode{461}
\prof{Professor Stashuk}
\title{
Design of a Hang-Glider Autonomous Controller\\
Interim Report
}
\date{\today}

\author{Andrew Cameron, 20252410}
\authorthree{Peter Lobsinger, 20195824}

\begin{document}

\maketitle

\section*{Introduction}
% This should introduce the topic in a manner understandable by anyone with a
% reasonable technical background (i.e. any 1st year engineering student should
% be able to understand the problem). Identify the need or problem that you want
% to address and why you want to address it. Indicate why and to whom the
% problem is important.  Background information should motivate the need for the
% design and show the nature, extent or significance of the problem you have
% defined.  This can be done, for example, by identifying inadequacies in
% existing designs, demonstrating an industry need for such an implementation,
% or reviewing existing research on the topic.  These three aspects
% (Introduction, Background and Problem Statement) can be organized in separate
% sections, or all together, etc. in order to suit the readability for your
% particular workshop.

\section*{Objectives}
% As with the Design Plan, objectives must be specific and demonstrated to be
% realistic.  The objectives should not have significantly changed from the
% Design Plan; however, you may want to rework them to ensure that they fit in
% with your current progress.

\section*{Methodology}
% At this point in the design process, your design methodology should be
% sufficiently developed so that the path to achieve the objectives is clear.
% Criteria (measurable qualities) and constraints (hard decisions) should be
% provided to guide the design process and motivate preferred solutions.  The
% extent of the use of criteria and constraints is project dependent.  The
% design group should provide sufficient information in this section to fully
% justify the methods used in the future of the workshop project.  If two or
% more potential solutions are being evaluated, then a decision matrix should be
% (eventually) utilized.

\section*{Timeline}
% The timeline provided with the Design Plan should be updated to indicate the
% current state of the project.  A concise description indicating how well the
% group is following their projected time-line should be placed in an appendix
% (along with the timeline).

\bibliography

\end{document}
